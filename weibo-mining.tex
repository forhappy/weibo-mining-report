%\documentclass[table, handout, pdftex]{beamer}
%\documentclass[CJK, table, handout]{beamer}
\documentclass[CJKutf8, table]{beamer}
%%%% theme used %%%%
\usetheme{Frankfurt}

%%%% import package %%%%
\usepackage{tikz}
\usepackage{CJKutf8}
\usepackage{graphicx}
\usepackage{pgf, pgfarrows, pgfnodes, pgfautomata, pgfheaps}
\usepackage{fancybox}
\usepackage{listings}
\usepackage{color}
\usepackage{caption}
\usepackage{hyperref}
\usepackage{textcomp}
\usepackage{enumerate}
\usepackage{courier}
\usepackage{listings}
\usepackage{tabularx}
\usepackage{wasysym}
\useoutertheme{progressbar}

\definecolor{listinggray}{gray}{0.9}
\definecolor{lbcolor}{rgb}{0.9,0.9,0.9}
\lstset{
	language=Java,
	captionpos=b,
	tabsize=4,
	frame=lines,
	keywordstyle=\color{blue},
	commentstyle=\color{darkgreen},
	stringstyle=\color{red},
	breaklines=true,
	showstringspaces=false,
	basicstyle=\footnotesize,
	emph={label}
	}

%\pgfdeclareimage[height=60pt]{logo}{ictlogo.png}
%\logo{\pgfuseimage{logo}\hspace{-2pt}\vspace{-8pt}}


\logo{\includegraphics[height=0.045\textwidth]{ictlogo.png}}

%%%% title page %%%%

\title[Weibo-Mining]{基于微博短文本特征词扩展与降维技术的\\网络热点分析}
\subtitle{}
\author[Fu H.P.]{傅海平\inst{1} \and 赵震\inst{2} \and 崔遥\inst{3} 
\and 王维\inst{4} \and 王宁宁\inst{5} \and 马苗\inst{6}}
\institute[ICT]{\textsc{\inst{1,2,3,4}计算技术研究所, \inst{5}城市规划发展研究中心, \inst{6}国家空间科学中心\\[5ex]}
\textbf{中国科学院研究生院, 北京海淀, 中国\\[3ex]}
\texttt{\inst{1}haiping.ict, \inst{4}wangwei881116}@gmail.com}
%\texttt{haipingf@gmail.com}
\date{\today}

% \AtBeginSection[]
% {
% \begin{frame}
%   \frametitle{Outline}
%   \tableofcontents[currentsection, currentsubsection, current]
% \end{frame}
% }

\AtBeginSection[]
{
\begin{frame}[shrink]
\begin{CJK}{UTF8}{gbsn}

	\frametitle{目录}
	\tableofcontents[%
 		currentsection, % causes all sections but the current to be shown in a semi-transparent way.
% 		currentsubsection, % causes all subsections but the current subsection in the current section to ...
% 		hideallsubsections, % causes all subsections to be hidden.
 		hideothersubsections, % causes the subsections of sections other than the current one to be hidden.
% 		part=, % part number causes the table of contents of part part number to be shown
%		pausesections, % causes a \pause command to be issued before each section. This is useful if you
% 		pausesubsections, %  causes a \pause command to be issued before each subsection.
% 		sections={ overlay specification },
	]
\end{CJK}
\end{frame}
}

%%%% begin document %%%%
\begin{document}

\begin{CJK}{UTF8}{gbsn}

\begin{frame}
\begin{CJK}{UTF8}{gbsn}
  \titlepage
\end{CJK}
\end{frame}

%%%%%%%%%%% 背景及目的 %%%%%%%%%%%%%%
\section{背景及目的}
\subsection{网络热点发现的兴起}
\begin{frame}
  \frametitle{Web2.0~时代的网络与个人}
\end{frame}

\subsection{网络热点挖掘的意义}
\begin{frame}
  \frametitle{网络热点挖掘的意义}
\end{frame}

%%%%%%%%%%% 研究现状 %%%%%%%%%%%%%%
\section{研究现状}
\subsection{国内研究现状}
\begin{frame}
  \frametitle{国内研究现状}
\end{frame}

\subsection{国外研究现状}
\begin{frame}
  \frametitle{国外研究现状}
\end{frame}

\subsection{小结}
\begin{frame}
  \frametitle{小结}
\end{frame}

%%%%%%%%%%% 设计方案 %%%%%%%%%%%%%%
\section{设计方案}
\subsection{微博预处理}
\begin{frame}
  \frametitle{微博预处理}
  \begin{itemize}
    \item<1- | alert@1>{分词: 拟采用计算所分词程序
      ~\href{http://ictclas.org/}{~ICTCLAS~}}
      \begin{itemize}
        \item[*]<2->{以下是广告时间\smiley}
        \item[*]<3- | alert@3>{ICTCLAS在国内973专家组组织的评测中活动获得了第一名,
          在第一届国际中文处理研究机构SigHan组织的评测中都获得了多项第一名。}
        \item[*]<4- | alert@4>{综合性能——ICTCLAS 2011分词速500KB/s左右,
          分词精度98.45\%,API不超过100KB,
          各种词典数据压缩后不到3M。}
        \item[*]<5- | alert@5>{全方位支持各种环境下的应用开发。}
      \end{itemize}
    \item<6- | alert@6>{去除停用词}
      \begin{itemize}
        \item[*]<7- | alert@7>{停用词初步定义为助词、介词、连词等虚词以及词语长度为
          1 的无实际含义的词}
        \item[*]<8- | alert@8>{停用词表:共收集了1928个停用词}
        \item[*]<9- | alert@9>{由于微博趋向于口语化,所以停用词需精心筛选}
      \end{itemize}
  \end{itemize}
\end{frame}

\subsection{文本建模}
\begin{frame}
  \frametitle{文本建模}
\end{frame}

\subsection{特征信息扩展}
\begin{frame}
  \frametitle{特征信息扩展}
\end{frame}

\subsection{降维}
\begin{frame}
  \frametitle{降维}
\end{frame}

\subsection{文本聚类}
\begin{frame}
  \frametitle{文本聚类}
\end{frame}

\subsection{系统结构图}
\begin{frame}
  \frametitle{系统结构图}
\end{frame}

%%%%%%%%%%% 系统实现 %%%%%%%%%%%%%%
\section{系统实现}

\subsection{微博分词}
\begin{frame}
  \frametitle{微博分词}
\end{frame}

\subsection{去除停用词}
\begin{frame}
  \frametitle{去除停用词}
\end{frame}

\subsection{特征信息扩展}
\begin{frame}
  \frametitle{特征信息扩展}
\end{frame}

\subsection{特征信息降维}
\begin{frame}
  \frametitle{特征信息降维}
\end{frame}

\subsection{特征词权重计算}
\begin{frame}
  \frametitle{特征词权重计算}
\end{frame}

\subsection{聚类}
\begin{frame}
  \frametitle{聚类}
\end{frame}

%%%%%%%%%%% 测试与结论 %%%%%%%%%%%%%%
\section{测试与结论}
\begin{frame}
  \frametitle{测试与结论}
\end{frame}
\end{CJK}
\end{document}
